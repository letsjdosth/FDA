\documentclass{article}
\usepackage{amsmath,amssymb,setspace,verbatim,graphicx,enumerate,enumitem}
\usepackage[top=1in,bottom=1in,left=1in,right=1in,head=0.5in,foot=0.5in]{geometry}
\usepackage{caption}
\usepackage{mathtools}
% \usepackage{subcaption}
% \usepackage{subfig}
% \usepackage{subfloat}
% \usepackage{tabularx}
\usepackage{mdframed}

\newenvironment{Rcode}% environment name 
{%begin code
    \begin{mdframed}
    \#R code
    \begin{small}
}
{%end code
    \end{small}
    \end{mdframed}
}

\newenvironment{console}% environment name 
{%begin code
    \begin{mdframed}
    \#Console
    \begin{small}
}
{%end code
    \end{small}
    \end{mdframed}
}

\begin{document}
\title{FDA Homework 3}
\author{Seokjun Choi}
\date{November 8th, 2019}
\maketitle

\section{Chapter 12}
\subsection{Problem 2}
\textbf{
Show that for every eigenvalue $\lambda$ of a bounded operator $L$,
we have $|\lambda|\leq||L||_{\mathcal{L}}$.
}

In problem's statement, $L$ is being assumed to be spectral decomposable 
(i.e. self-adjoint, compact(or more strongly, Hilbert-Schmidt) operator), so I will work with these assumptions.

Firstly I claim that $||L||$ or $-||L||$ is eigenvalue of $L$.
Without loss of generality, assume the former case and denote $\lambda_1=||L||=\sup\{<Lf,f>:||f||=1\}$.
(the proof for $||L||=-\lambda_1$ case is similar to following argument.)

Let $\{f_n\}\in\mathcal{H}$ such that $||f_n||=1$, $<Tf_n,f_n>\rightarrow\lambda_1$ 
and $Tf_n\rightarrow g$ for some $g\in\mathcal{H}$.
(such $\{f_n\},g$ exist because $L$ is compact and $\mathcal{H}$ is complete.)
Then
\[||Lf_n-\lambda_1 f_n||^2=||Lf_n||^2-2\lambda_1<Lf_n,f_n>+\lambda_1^2||f_n||^2\]
\[\leq ||L||^2||f_n||^2-2\lambda_1<Lf_n,f_n>+\lambda_1^2||f_n||^2\]
\[\leq \lambda_1^2-2\lambda_1<Lf_n,f_n>+\lambda_1^2\]
\[\leq 2\lambda_1^2-2\lambda_1^2\rightarrow 0\]
So $Lf_n\rightarrow g$, $\lambda_1 f_n\rightarrow g$, and under continuity of $L$ from the assumptions,
we get $\lambda_1 g = Lg$. And we also verify $g\neq0$ because if 0, it becomes $\lambda_1=||L||=0$, contradiction.
thus $\lambda_1$ is eigenvalue of $L$.


Next, I claim one more thing that above $\lambda_1=\max{|\lambda|}$ over all $\lambda$s which are eigenvalues of $L$.
Without loss of generality, assume as above and the case all eigenvalues of $L$ is nonnegative. 
(if not, change sign of it and its pair eigenfunction together.)
Assume the claim is false, then there are $\lambda^*$ and the pair eigenfuction $v^*$ whose norm is 1.
Then, $\lambda^*=||Lv^*1||>\lambda_1=||L||=\sup_{||v||=1}||Lv||$, we have contradiction (to sup and definition of operator norm).
So, all eigenvalues of $L$ are smaller then $\lambda_1$, and it is what we want,
$|\lambda|\leq||L||=\lambda_1$.




\subsection{Problem 5}
\textbf{
Assume that $X_1,...,X_N$ are iid element of $L^2[0,1]$ with
$E||X_n||^4<\infty$ and whose first $p$ eigenvalue are distinct.
Prove that
\[|N<\hat{v}_j-v_j,v_j>|=O_P(1) \text{ for } j=1,...,p\]
Why is this a seemingly unusual convergence rate? 
(Hint: \(|<\hat{v}_j-v_j,v_j>|=\frac{1}{2}||\hat{v}_j-v_j||\))
}

With hint of the problem, I'll show that
\(N||\hat{v_j}-v_j||^2\) is bounded in probability sense, or $O_P(1)$.\\
(constant $1/2$ does not matter in this context.)

Then by theorem 12.2.1 on our book, under our assumptions, we know that
\[\limsup_{N\rightarrow \infty} NE[||\hat{v}_j-v_j||^2]<C\]
for some $C$ and $j=1,...,p$. 
On other hand, by Chebyshev's inequality, for all $\alpha>0$,
\[Pr(N||(\hat{v_j}-v_j)||^2>\alpha)<\frac{NE[||\hat{v}_j-v_j||^2]}{\alpha}\]
The result of above theorem says that right-hand side is bounded by $C/\alpha$ in probability sense for all $\alpha>0$.
Then using definition of boundedness in probability to left-hand side,
we get \(N||\hat{v_j}-v_j||^2\) is also bounded in probability, which we want.

Since this proof (and the base theorem) depend on a strong condition that whole points of function are completely observable, 
the inner product term (view as 'weak distance' heuristically) is bounded of rate $O(N^{-1})$,
unusual(-ly good) rate comparing to ordinary parametric rate $O(N^{-1/2})$. 


\subsection{Problem 6}
\textbf{
Prove Theorem 12.1.3:
Let $x,y\in\mathcal{H}$. Then
\(||x\otimes y||_{\mathcal{H}\otimes\mathcal{H}}=
||<y,.>x||_{\mathcal{S}}\)
}

When we view tensor as operator, \(x\otimes y(.)=<y,.>x\) holds. 
Using this fact, when $\{e_i\}$ are orthonormal basis of $\mathcal{H}$,
by the definition of Hilbert-Schmidt norm equipped on $\mathcal{S}$, we are able to write
\[||<y,.>x||^2_{\mathcal{S}}=\sum_{i=1}^{\infty}||(<y,z>x)e_i||^2 \text{ for } \forall z\in\mathcal{H}\]
then by Parseval's identity,
\[=||<y,z>x||^2_{\mathcal{H}} \text{ for } \forall z\in\mathcal{H}\]
then by above relation between inner product and tensor product, and since $z$ is arbitrary,
\[=||x\otimes y||^2_{\mathcal{H}\otimes\mathcal{H}}\]


\subsection{Problem 7}
\textbf{
Suppose that the data ${X_n(t):t\in[0,1], 1\leq n\leq N}$
are expressed using an orthonormal basis $e_1,...,e_J$:
\[X_n(t)=\sum_{j=1}^J x_{n_j}e_j(t)\]
In this case, the EFPC's, $\hat{v}_i(t)$ can also be expressed as
\[\hat{v}_i(t)=\sum_{j=1}^J \hat{v}_{ij}e_j(t)\]
Explain how to obtain the coefficient $\hat{v}_{ij}$ from the $x_{nj}$. 
Justify your answer.
}

I cannot ensure the purpose of this problem. So I will give some outline to get $\hat{v}$.

For notational convenience, I'll omit hat-expression (except some variable) despite almost all variables are sample value.
\begin{enumerate}
\item Fit to N function-objects $X_1,...,X_N$ using basis $\{e_j\}$ in (normally) $L^2$ with data points $x_{nj}$,
    and get $X_n(t)=\sum_m c_{nm}B_m(t)$ form where $B_m$ is m-th basis and t becomes data points of each curve. \\
    In practice, do some transformation corresponding to chosen basis 
    (for example, if you choose Fourier basis, then get Fourier series expression.)
     and keep the coefficients $c_{nm}$ of each basis and of each curves as form of matrix.
\item Get sample mean function $\hat{\mu}(t)=N^{-1}\sum_{n=1}^NX_n(t)$.
    we get it easily by just calculating means over each coefficients of each basis respectively and taking sum,
    so $\hat{\mu}(t)=\sum_m(N^{-1}\sum_nc_{nm})B_m(t)$. \\
    In practice, we only need to keep $\bar{c}_m=N^{-1}\sum_nc_{nm}$.
\item Get sample covariance operator (if need, adjust mean to 0.) $\hat{c}(t,s)=N^{-1}\sum_{n=1}^N(X_n(t)-\hat{\mu}(t))(X_n(s)-\hat{\mu}(s))$.\\
    In practice, calculate $\tilde{c}_{nm}=c_{nm}-\bar{c}_{m}$ and get $\tilde{c}_{m_1}^{T}\tilde{c}_{m_2}$,
    then the ($m_1m_2$)-th element becomes the coefficient of $B_{m_1}(t)B_{m_2}(t)$.
\item Find eigenpairs, $\lambda_j v_j(t) = \int\hat{c}(t,s)v_j(s)ds$. \\ 
    In practice, find $\lambda_j$s and $v_j$s by solving $\lambda_j\int v_j(t)B_{m_1}(t)dt = \int \int \hat{c}(t,s)v_j(s)B_{m_1}(t)ds dt$.
    In detail, we replace integral with sum, $\lambda_j v_j(t) = \sum_{s} \hat{c}(t,s)v_s$, so solve
    $\lambda_jv_{jm_1}=\sum_{m_3}\tilde{c}_{m_1}^{T}\tilde{c}_{m_3} v_{jm_3}$ for all $j$.
\end{enumerate}
At last step, (replace notation $j$ to $i$ and now use $j$ for index of basis) we get $\hat{v}_{ij}$ values which we want.


\subsection{Problem 12}
\textbf{
Under the same assumptions as in Problem 12.8.5,
shows that, for $j\neq k$ and $1\leq j \leq p$,
\[<\hat{v}_j-v_j, v_k>=\frac{<\hat{C}-C,\hat{v}_j \otimes v_k>}{\hat{\lambda}_j-\lambda_k}\]
What can you conclude about the asymptotic distribution of $N^{-1/2}<\hat{v}_j-v_j,v_k>$?
}

By theorem 12.3.2 on our book, under our assumptions we know that 
\[N^{1/2}(\hat{v}_j-v_j)=\sum_{i\neq j}\frac{1}{\lambda_j-\lambda_i}<\sqrt{N}(\hat{C}-C), v_i\otimes v_j>v_i+o_P(1)\]
Take both side to $<., v_k>$ and neglect ignorable term, then
\[N^{1/2}<\hat{v}_j-v_j,v_k>=\sum_{i\neq j}\frac{1}{\lambda_j-\lambda_i}<\sqrt{N}(\hat{C}-C), v_i\otimes v_j><v_i,v_k>\]
then, the last inner product term becomes 1 only $i=k$, and otherwise 0. So we can rewrite it without summation symbol as 
\[N^{1/2}<\hat{v}_j-v_j,v_k>=\frac{\sqrt{N}<\hat{C}-C, v_k\otimes v_j>}{\lambda_j-\lambda_k}\]
multiply $1/\sqrt{N}$ to both sides.
And, because $\hat{\lambda}_j\rightarrow\lambda_j$ and $\hat{v}_j\rightarrow v_j$ as $N\rightarrow \infty$
in probability sense.
(For showing eigenfunction part convergence in detail, divide the form of problem 5's result by $N$ and observe it. 
And for eigenvalue part convergence, do like procedure of solving problem 5 using the result of eigenvalue part of theorem 12.2.1 and applying Chebyshev's inequality.)\\ 
So we can replace $\lambda_j, v_j$ with $\hat{\lambda}_j,\hat{v}_j$ with adding only ignorable terms in right-hand side,
and if we disregard them, we get
\[<\hat{v}_j-v_j, v_k>=\frac{<\hat{C}-C,\hat{v}_j \otimes v_k>}{\hat{\lambda}_j-\lambda_k}\]
which we want. 

(comment: Can I commute tensor product term in inner product? Intuitively, in general, may be No. But in this case,
thinking only above equation and view $C,\hat{C},v_k\otimes\hat{v}_j$ 
as operator of $\mathcal{H}_1\times\mathcal{H}_2\rightarrow\mathcal{H}$ 
where $\mathcal{H}=\mathcal{H}_1=\mathcal{H}_2$,
let's consider the domain-permuted(?) operators of $C,\hat{C},\hat{v}_j\otimes v_k$ 
such that $\mathcal{H}_2\times\mathcal{H}_1\rightarrow\mathcal{H}$.
then, $C, \hat{C}$ is invariant because of their definition, so are LHS's $v_j, \hat{v}_j, v_k$ and RHS's $\hat{\lambda}_j,\lambda_k$.
And only last thing is changed to $\hat{v}_j \otimes v_k$. So at above equation, I may commute with tensor product without any other adjustment in our special context.)

For finding distribution, using this result and apply CLT to $\sqrt{N}(\hat{C}-C)$ (like theorem 12.3.1), or taking the result of corollary12.3.1 directly, 
we get the asymptotic distribution of \(N^{-1/2}<\hat{v}_j-v_j,v_k>\) as
\[N(0, N^{-1/2}<C_j,v_j \otimes v_k>)\]
where $C_j$ is of the corollary's, i.e.
\[C_j=\sum_{i\neq j}\sum_{l\neq j}\frac{<\Gamma,v_i\otimes v_j\otimes v_l\otimes v_j>}
{(\lambda_j-\lambda_i)(\lambda_j-\lambda_l)}(v_i \otimes v_l)\]
\[\Gamma=E[[(X-\mu)\otimes(X-\mu)-C]\otimes[(X-\mu)\otimes(X-\mu)-C]]\]
and $\mu$ is mean function of $X$.


\end{document}