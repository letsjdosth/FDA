\documentclass{article}
\usepackage{amsmath,amssymb,setspace,verbatim,graphicx,enumerate,enumitem}
\usepackage[top=1in,bottom=1in,left=1in,right=1in,nohead,nofoot]{geometry}
\usepackage{caption}
\usepackage{mathtools}
% \usepackage{subcaption}
% \usepackage{subfig}
% \usepackage{subfloat}
% \usepackage{tabularx}
\usepackage{mdframed}

\newenvironment{Rcode}% environment name 
{%begin code
    \begin{mdframed}
    \#R code
    \begin{small}
}
{%end code
    \end{small}
    \end{mdframed}
}

\newenvironment{console}% environment name 
{%begin code
    \begin{mdframed}
    \#Console
    \begin{small}
}
{%end code
    \end{small}
    \end{mdframed}
}

\begin{document}
\title{FDA Homework 3}
\author{Seokjun Choi}
\date{November 8th, 2019}
\maketitle

\section{Chapter 12}
\subsection{Problem 2}
\textbf{
Show that for every eigenvalue $\lambda$ of a bounded operator $L$,
we have $|\lambda|\leq||L||_{\mathcal{L}}$.
}

In problem's statement, $L$ is being assumed that it can be spectral decomposable 
(i.e. self-adjoint, compact(or, Hilbert-Schmidt) operator), so I will work with these assumptions.

Firstly I claim that $||L||$ or $-||L||$ is eigenvalue of $L$.
Without loss of generality, assume first case and denote $\lambda_1=||L||=\sup\{<Lf,f>:||f||=1\}$.
Let $\{f_n\}\in\mathcal{H}$ such that $||f_n||=1$, $<Tf_n,f_n>\rightarrow\lambda_1$ 
and $Tf_n\rightarrow g$ for some $g\in\mathcal{H}$.
(such $\{f_n\},g$ exist because $L$ is compact and $\mathcal{H}$ is complete.)
Then
\[||Lf_n-\lambda_1 f_n||^2=||Lf_n||^2-2\lambda_1<Lf_n,f_n>+\lambda_1^2||f_n||^2\]
\[\leq ||L||^2||f_n||^2-2\lambda_1<Lf_n,f_n>+\lambda_1^2||f_n||^2\]
\[\leq \lambda_1^2-2\lambda_1<Lf_n,f_n>+\lambda_1^2\]
\[\leq 2\lambda_1^2-2\lambda_1^2\rightarrow 0\]
So $Lf_n\rightarrow g$, $\lambda_1 f_n\rightarrow g$, and under continuity of $L$ from the assumptions,
we get $\lambda_1 g = Lg$. And we also verify $g\neq0$ because if 0, it becomes $\lambda_1=||L||=0$, contradiction.
thus $\lambda_1$ is eigenvalue of $L$.
(The proof for $||L||=-\lambda_1$ case is similar.)

Next I claim one more thing that above $\lambda_1=\max{|\lambda|}$ over all $\lambda$s which are eigenvalues of $L$.
Without loss of generality, consider only the case all eigenvalues of $L$ is nonnegative. 
(if not, change sign of it and its pair eigenfunction together.)
Assume the claim is false, then there are $\lambda^*$ and the pair eigenfuction $v^*$ whose norm is 1.
Then, $Lv^*=\lambda^*>\lambda_1=||L||=\sup_{||v||=1}||Lv||$, we have contradiction.
So, all eigenvalues of $L$ are smaller then $\lambda_1$, and it is what we want,
$|\lambda|\leq||L||=\lambda_1$.




\subsection{Problem 5}
\textbf{
Assume that $X_1,...,X_N$ are iid element of $L^2[0,1]$ with
$E||X_n||^4<\infty$ and whose first $p$ eigenvalue are distinct.
Prove that
\[|N<\hat{v}_j-v_j,v_j>|=O_P(1) \text{ for } j=1,...,p\]
Why is this a seemingly unusual convergence rate? 
(Hint: \(|<\hat{v}_j-v_j,v_j>|=\frac{1}{2}||\hat{v}_j-v_j||\))
}

With hint of the problem, I'll show that
\(N||\hat{v_j}-v_j||^2\) is bounded in probability sense, or $O_P(1)$.\\
(constant $1/2$ does not matter in this context.)

Then by theorem 12.2.1 on our book, under our assumptions, we know that
\[\limsup_{N\rightarrow \infty} NE[||\hat{v}_j-v_j||^2]<C\]
for some $C$ and $j=1,...,p$. 
On other hand, by Chebyshev's inequality,
\[Pr(N||(\hat{v_j}-v_j)||^2>\alpha)<\frac{NE[||\hat{v}_j-v_j||^2]}{\alpha}\]
The result of above theorem says that right-hand side is bounded in probability sense for all $\alpha>0$.
Then using definition of boundedness in probability to left-hand side,
we get \(N||\hat{v_j}-v_j||^2\) is bounded in probability, which we want.



\subsection{Problem 6}
\textbf{
Prove Theorem 12.1.3:
Let $x,y\in\mathcal{H}$. Then
\(||x\otimes y||_{\mathcal{H}\otimes\mathcal{H}}=
||<y,.>x||_{\mathcal{S}}\)
}

When we view tensor as operator, \(x\otimes y(.)=<y,.>x\) holds. 
Using this fact, when $\{e_i\}$ are orthonormal basis of $\mathcal{H}$,
by the definition of Hilbert-Schmidt norm equipped on $\mathcal{S}$,
\[||<y,.>x||^2_{\mathcal{S}}=\sum_{i=1}^{\infty}||(<y,z>x)e_i||^2 \text{ for } \forall z\in\mathcal{H}\]
then by Parseval's identity,
\[=||<y,z>x||^2_{\mathcal{H}} \text{ for } \forall z\in\mathcal{H}\]
then by above fact and $z$ is arbitrary,
\[=||x\otimes y||^2_{\mathcal{H}\otimes\mathcal{H}}\]


\subsection{Problem 7}
\textbf{
Suppose that the data ${X_n(t):t\in[0,1], 1\leq n\leq N}$
are expressed using an orthonormal basis $e_1,...,e_J$:
\[X_n(t)=\sum_{j=1}^J x_{n_j}e_j(t)\]
In this case, the EFPC's, $\hat{v}_i(t)$ can also be expressed as
\[\hat{v}_i(t)=\sum_{j=1}^J \hat{v}_{ij}e_j(t)\]
Explain how to obtain the coefficient $\hat{v}_{ij}$ from the $x_{nj}$. 
Justify your answer.
}

I cannot ensure the intention of this problem.


\subsection{Problem 12}
\textbf{
Under the same assumptions as in Problem 12.8.5,
shows that, for $j\neq k$ and $1\leq j \leq p$,
\[<\hat{v}_j-v_j, v_k>=\frac{<\hat{C}-C,\hat{v}_j \otimes v_k>}{\hat{\lambda}_j-\lambda_k}\]
}

By theorem 12.3.2 on our book, under our assumptions we know that 
\[N^{1/2}(\hat{v}_j-v_j)=\sum_{i\neq j}\frac{1}{\lambda_j-\lambda_i}<\sqrt{N}(\hat{C}-C), v_i\otimes v_j>v_i+o_P(1)\]
Take both side to $<., v_k>$ and neglect ignorable term, then
\[N^{1/2}<\hat{v}_j-v_j,v_k>=\sum_{i\neq j}\frac{1}{\lambda_j-\lambda_i}<\sqrt{N}(\hat{C}-C), v_i\otimes v_j><v_i,v_k>\]
then, the last inner product term becomes 1 only $i=k$, and otherwise 0. So we can rewrite it without summation symbol as 
\[N^{1/2}<\hat{v}_j-v_j,v_k>=\frac{\sqrt{N}<(\hat{C}-C), v_k\otimes v_j>}{\lambda_j-\lambda_k}\]
multiply $\sqrt{N}$ to both sides.
And, because $\hat{\lambda}_j\rightarrow\lambda_j$ and $\hat{v}_j\rightarrow v_j$ as $N\rightarrow \infty$
in probability sense by corollary 12.3.1.\\
(Or for eigenfunction part convergence, using problem 5's result considering the form dividing problem's expression by $N$. 
And for eigenvalue part convergence, do like procedure of solving problem 5 using the eigenvalue part of theorem 12.2.1 and Chebyshev's inequality.)\\ 
So we can replace $\lambda_j, v_j$ with $\hat{\lambda}_j,\hat{v}_j$ with adding only ignorable terms in right-hand side,
and if we disregard them, we get
\[<\hat{v}_j-v_j, v_k>=\frac{<\hat{C}-C,\hat{v}_j \otimes v_k>}{\hat{\lambda}_j-\lambda_k}\]
which we want


\end{document}