\documentclass{article}
\usepackage{amsmath,amssymb,setspace,verbatim,graphicx,enumerate,enumitem}
\usepackage[top=1in,bottom=1in,left=1in,right=1in,nohead,nofoot]{geometry}
\usepackage{caption}
% \usepackage{subcaption}
% \usepackage{subfig}
% \usepackage{subfloat}
% \usepackage{tabularx}
\usepackage{mdframed}

\newenvironment{Rcode}% environment name 
{%begin code
    \begin{mdframed}
    \#R code
    \begin{small}
}
{%end code
    \end{small}
    \end{mdframed}
}

\newenvironment{console}% environment name 
{%begin code
    \begin{mdframed}
    \#Console
    \begin{small}
}
{%end code
    \end{small}
    \end{mdframed}
}

\begin{document}
\title{FDA Homework 2}
\author{Seokjun Choi}
\date{October 18, 2019}
\maketitle

\section{Chapter 10}
\subsection{Problem 2}

\textbf{
Show that in any inner product space, the function $y\rightarrow<x,y>$ is continuous where x is arbitary element of that inner product space.
}

Let $\mathcal{H}$ be an inner product space, \(\{f_n\}\) be a sequence in $\mathcal{H}$
such that converges to $f\in\mathcal{H}$ in norm sense. \\
For $x\in\mathcal{H}$, consider below relation.
\[|<x, f_n>-<x,f>|^2 = |<x, f_n-f>|^2 \leq ||x||^2||f_n-f||^2\]
Last inequality comse from Cauchy-Schwartz inequality. Then when $n\rightarrow\infty$, by our setting $||f_n-f||\rightarrow0$, so
\[lim_{n\rightarrow\infty}|<x, f_n>-<x,f>|^2\leq 0\]
Then
\[lim_{n\rightarrow\infty}<x, f_n>-<x,f>= 0\]
Thus, \(lim_{n\rightarrow\infty}<x, f_n> = <x,f>\) and the inner product operator is preserve the limit.
It is equivalent statement that inner product operator is continuous.


\subsection{Problem 6}
\textbf{
    Suppose $\{e_j,j\geq 1\}$ is a complete orthonormal sequence in a Hilbert space. 
    Show that if $\{f_j,j\geq 1\}$ is an orthonormal sequence satisfying
    \[\sum_{j=1}^\infty ||e_j-f_j||^2<1\]
    then $\{f_j,j\geq 1\}$ is also complete.
}

Firstly, I claim that $e_j$ and $f_j$ are not orthorgonal. \\
Consider a simple case that only one component of fixed $j$ index are different $e_j$ from $f_j$. then,
If $e_j, f_j$ are orthonormal, the by Pythagorean theorem, $||e_j-f_j||^2=||e_j||^2+||f_j||^2=1+1=2$
since $e_j$ and $f_j$ are unit elements. But under our assumption
\[||e_j-f_j||^2=<e_j-f_j, e_j-f_j>=<e_j, e_j>-<e_j,f_j>-<f_j,e_j>+<f_j,f_j>\]
\[=||e_j||^2+||f_j||^2-(<e_j,f_j>+<f_j,e_j>) = 2-(<e_j,f_j>+<f_j,e_j>) < 1\]
so in our case, $||e_j-f_j||^2$ always cannot have the value 2, and 
\[<e_j,f_j>+<f_j,e_j>=<e_j,f_j>+\overline{<e_j,f_j>}=2Re(<e_j,f_j>)\] cannot become 0,
neither can $<e_j,f_j>$.
Both viewing norm value and inner product value show that they are not orthogonal.
Thus the claim is proved for simple case.

Next, consider many j-th components can be different $e_j$ from $f_j$. But above claim still holds, because
\(\sum_{j=1}^\infty ||e_j-f_j||^2<1\) and all norm values must be nonnegative,
none of j-th component of $||e_j-f_j||$ can be greater then 1 like the simple case, 
and $<e_j,f_j>+<f_j,e_j>=2Re(<e_j,f_j>)$ cannot be 0 for all $j$.
So above claim is proved for whole case.


Then from the result of claim, if we express $f_j$ as the linear combination of $\{e_i\}$, 
then $f_j$ always has $e_j$ component with nonzero coefficient.
(More precisely, use Gram-Schmidt process, or just project $f_j$ onto $e_j$.
then since $f_j$ is not orthogonal to $e_j$, by property of separable Hilbert space's orthonormal basis,
we get $e_j$ component of nonzero coefficient.)

So, if we express whole space as \(\mathcal{H}=span\{e_1,e_2,...,e_j,...\}\), then
we can also re-express $\mathcal{H}$ as \(\mathcal{H}=span\{e_1,e_2,...,f_j,...\}\).
Then for each $j$, we replace $e_j$ to $f_j$ inductively from $j=1$ to $\infty$ for above span expression.
Then consquently we get \(\mathcal{H}=span\{f_1,f_2,...,f_j,...\}\) for $\{f_j, j\geq 1\}$, it's what we want.




\subsection{Problem 10}
\textbf{
    Suppose $\{e_j,j>=1\}$ and $\{f_i,i>=1\}$ are orthonormal bases in $\mathcal{H}$.
    Show that for any Hilbert-Schmidt operators $\Psi,\Phi$
    \[\sum_{i=1}^{\infty} <\Psi(f_i), \Phi(f_i)> = \sum_{j=1}^{\infty} <\Psi(e_j),\Phi(e_j)>\]
}

Firstly note that $f_i=\sum_{j=1}^\infty<f_i,e_j>e_j$, 
and since $\Phi$ are Hilbert-Schmidt, there are adjoint operator  $\Phi^*$.
Using these facts,
\[\sum_{i=1}^{\infty} <\Psi(f_i), \Phi(f_i)>
= \sum_{i=1}^{\infty} <\Phi^*\Psi(f_i), f_i> 
= \sum_{i=1}^\infty <\Phi^*\Psi\sum_{j=1}^\infty <f_i,e_j>e_j, \sum_{k=1}^\infty <f_i,e_k>e_j>\]
then
\[=\sum _{i=1}\sum_{j=1}\sum_{k=1}<f_i,e_j>\overline{<f_i,e_k>}<\Phi^*\Psi(e_j),e_k>
=\sum _{i=1}\sum_{j=1}\sum_{k=1}<f_i,e_j><e_k,f_i><\Phi^*\Psi(e_j),e_k>\]
then when $j\neq k$, the term becomes 0.(why?)
so, only $j=k$ cases remain, and we rewrite above equation as
\[=\sum_{i=1}\sum_{j=1} <f_i,e_j><e_j,f_i><\Phi^*\Psi(e_j),e_j>\]
Since the operaters are Hilbert-Schmidt, the value of absolute summation is bounded
and we can interchange the summation order. then
\[=\sum_{j=1}\sum_{i=1} <f_i,e_j><e_j,f_i><\Phi^*\Psi(e_j),e_j>
=\sum_{j=1}\sum_{i=1}|<f_i,e_j>|^2<\Phi^*\Psi(e_j),e_j>\]
\[=\sum_{j=1} ||e_j||^2<\Phi^*\Psi(e_j),e_j>
=\sum_{j=1} <\Phi^*\Psi(e_j),e_j>
\]
At last part, I use the relation that $||e_j||^2=\sum_{i=1}|<f_i,e_j>|^2 = 1$  
from the assumption that $\{f_i\}$ is in orthornormal set.
\[=\sum_{j=1} <\Psi(e_j),\Phi(e_j)>
\]



\subsection{Problem 12}
\textbf{
    Show that if $L$ is bounded then $L^*$ is also bounded, and
    \[||L^*||_{\mathcal{L}}=||L||_{\mathcal{L}}, \quad  
    ||L^*L||_{\mathcal{L}}=||L||^2_{\mathcal{L}}
    \]
}

Firstly, I construct a lemma.
\[||L||_{\mathcal{L}} = \sup\{|<Lx,y>|:||x||\leq 1, ||y||\leq 1\}\]
For proof, (1) $\geq$ direction is simple consquence of Cauchy-Schwartz inequality.
\(|<Lx,y>|\leq ||Lx||||y|| \leq ||Lx||\) for $||x||\leq 1$. Take supremum(within $||x||\leq 1$) both side.
(2) To get $\leq$ direction, for $x\in \mathcal{H}, ||x||\leq 1$,
let $x'=x/||x||, y'=Lx/||Lx||$. Then if $\sup|<Lx,y>|= M$ for some $M\in\mathcal{R}$, then $|<Lx',y'>|\leq M$ by assumption, and observe that
\(|<Lx',y'>|=|<\frac{Lx}{||x||},\frac{Lx}{||Lx||}>=\frac{||Lx||^2}{||x||||Lx||}=\frac{||Lx||}{||x||}\), so
$||Lx||\leq M||x||$. Take supremum(within $||x||\leq 1$) both side.

Note that the argument which gives supremum value always satisfies $||x||=1$.

Let's start our main goal.
Let $x\in \mathcal{H} \text{ such that } ||x||\leq 1$, be argument of realizing supremum of definition of operator norm, \(||L||_{L}=||Lx||\))

For boundedness and first identity, consider
\[||L||_\mathcal{{L}}=\sup\{|<Lx, y>| : :||x||\leq 1, ||y||\leq 1\}\]
\[=\sup\{|<x, L^*y>| : :||x||\leq 1, ||y||\leq 1\} = ||L^*||\]
So if $L$ is bounded($||L||_\mathcal{L}<\infty$), then $L^*$ is also bounded($||L^*||_\mathcal{L}<\infty$).

For second identity, consider
\[||L||_\mathcal{{L}}=\sup\{|<Lx, Ly/||Ly||>| :||x||\leq 1, ||y||\leq 1\}\]
\[=\sup\{|\frac{<L^*Lx, y>}{\overline{||Ly||}}| : ||x||\leq 1, ||y||\leq 1\}\]
\[=\frac{1}{||L||_{\mathcal{L}}}\sup\{|<L^*Lx, y>| : ||x||\leq 1, ||y||\leq 1\}=\frac{||L^*L||_{\mathcal{L}}}{||L||_{\mathcal{L}}}\]
At third line, I use the definition of operater norm. Multiply $||L||_{\mathcal{L}}$ both side.
\end{document}

