\documentclass{article}
\usepackage{amsmath,amssymb,setspace,verbatim,graphicx,enumerate,enumitem}
\usepackage[top=1in,bottom=1in,left=1in,right=1in,nohead,nofoot]{geometry}
\usepackage{caption}
% \usepackage{subcaption}
% \usepackage{subfig}
% \usepackage{subfloat}
% \usepackage{tabularx}
\usepackage{mdframed}

\newenvironment{Rcode}% environment name 
{%begin code
    \begin{mdframed}
    \#R code
    \begin{small}
}
{%end code
    \end{small}
    \end{mdframed}
}

\newenvironment{console}% environment name 
{%begin code
    \begin{mdframed}
    \#Console
    \begin{small}
}
{%end code
    \end{small}
    \end{mdframed}
}

\begin{document}
\title{FDA Homework 2}
\author{Seokjun Choi}
\date{October 18, 2019}
\maketitle

\section{Chapter 10}
\subsection{Problem 2}

\textbf{
Show that in any inner product space, the function $y\rightarrow<x,y>$ is continuous where x is arbitary element of that inner product space.
}

Let $\mathcal{H}$ be an inner product space, \(\{f_n\}\) be a sequence in $\mathcal{H}$
such that converges to $f\in\mathcal{H}$ in norm sense. \\
For $x\in\mathcal{H}$, consider below relation.
\[|<x, f_n>-<x,f>|^2 = |<x, f_n-f>|^2 \leq ||x||^2||f_n-f||^2\]
Last inequality comse from Cauchy-Schwartz inequality. Then when $n\rightarrow\infty$, by our setting $||f_n-f||\rightarrow0$, so
\[lim_{n\rightarrow\infty}|<x, f_n>-<x,f>|^2\leq 0\]
Then
\[lim_{n\rightarrow\infty}<x, f_n>-<x,f>= 0\]
Thus, \(lim_{n\rightarrow\infty}<x, f_n> = <x,f>\) and the inner product operator is preserve the limit.
It is equivalent statement that inner product operator is continuous.


\subsection{Problem 6}
\textbf{
    Suppose $\{e_j,j>=1\}$ is a complete orthonormal sequence in a Hilbert space. 
    Show that if $\{f_j,j>=1\}$ is an orthonormal sequence satisfying
    \[\sum_{j=1}^\infty ||e_j-f_j||^2<1\]
    then $\{f_j,j>=1\}$ is also complete.
}

\subsection{Problem 10}
\textbf{
    Suppose $\{e_j,j>=1\}$ and $\{f_i,i>=1\}$ are orthonormal bases in $\mathcal{H}$.
    Show that for any Hilbert-Schmidt operators $\Psi,\Phi$
    \[\sum_{i=1}^{\infty} <\Psi(f_i), \Phi(f_i)> = \sum_{j=1}^{\infty} <\Psi(e_j),\Phi(e_j)>\]
}

Firstly note that $f_i=\sum_{j=1}^\infty<f_i,e_j>e_j$, 
and since $\Phi$ are Hilbert-Schmidt, there are adjoint operator  $\Phi^*$.
Using these facts,
\[\sum_{i=1}^{\infty} <\Psi(f_i), \Phi(f_i)>
= \sum_{i=1}^{\infty} <\Phi^*\Psi(f_i), f_i> 
= \sum_{i=1}^\infty <\Phi^*\Psi\sum_{j=1}^\infty (f_i,e_j)e_j, \sum_{k=1}^\infty (f_k,e_k)e_j>\]
then
\[=\sum _{i=1}\sum_{j=1}\sum_{k=1}<f_i,e_j>\bar{<f_i,e_k>}<\Phi^*\Psi(e_j),e_k>
=\sum _{i=1}\sum_{j=1}\sum_{k=1}<f_i,e_j><e_k,f_i><\Phi^*\Psi(e_j),e_k>\]
then when $j\neq k$, the term becomes 0.(why?)
so, only $j=k$ cases remain, so we rewrite above equation as
\[=\sum_{i=1}\sum_{j=1} <f_i,e_j><e_j,f_i><\Phi^*\Psi(e_j),e_j>\]
Since the operaters are Hilbert-Schmidt, the value of absolute summation is bounded,
and we can interchange the summation order. then
\[=\sum_{j=1}\sum_{i=1} <f_i,e_j><e_j,f_i><\Phi^*\Psi(e_j),e_j>
=\sum_{j=1}\sum_{i=1}|<f_i,e_j>|^2<\Phi^*\Psi(e_j),e_j>\]
\[=\sum_{j=1} <\Phi^*\Psi(e_j),e_j>
=\sum_{j=1} <\Phi^*\Psi(e_j),e_j>
\]
$\sum_{i=1}|<f_i,e_j>|^2 = 1$ since it coincide the definition of norm square, 
and each element is in orthornormal set.
\[=\sum_{j=1} <\Psi(e_j),\Phi(e_j)>
    \]



\subsection{Problem 12}
\textbf{
    Show that if $L$ is bounded then $L^*$ is also bounded, and
    \[||L^*||_{\mathcal{L}}=||L||_{\mathcal{L}}, \quad  
    ||L^*L||_{\mathcal{L}}=||L||^2_{\mathcal{L}}
    \]
}


\end{document}