\ifcase 0  % choose 0=slides, 1=article, 2=refart
   \documentclass[aspectratio=169,ignorenonframetext,9pt]{beamer}
\or\documentclass[a4paper,11pt]{article}
   \usepackage{url,beamerarticle}
\or\documentclass[a4paper,11pt]{refart}
   \let\example\relax
   \usepackage{url,beamerarticle}
\fi

\ifcase 0  % choose a theme like these
    % \usetheme{boxes}
    \usetheme{Boadilla}
    % \usetheme{Goettingen}% I recommend
    % \usetheme{Singapore}
    % \usetheme{Pittsburgh}
    % \usetheme{Madrid}
    % \usetheme{Warsaw} % common choice, but often poor
\fi

\usepackage{graphicx,pgfplots,parskip}
\usepackage{amsmath,amsfonts,amssymb,amsthm,epsfig,epstopdf,url,array}



\theoremstyle{plain}
\newtheorem{thm}{Theorem}[section]
\newtheorem{lem}[thm]{Lemma}
\newtheorem{prop}[thm]{Proposition}

\theoremstyle{definition}
\newtheorem{defn}{Definition}[section]
\newtheorem{conj}{Conjecture}[section]
\newtheorem{exmp}{Example}[section]


\title{Functional depths and related topics} 
\author{Choi Seokjun}
\date{11 Dec. 2019}


\begin{document}

\begin{frame}
\maketitle
\end{frame}


% \begin{abstract}
% This abstract, being outside the frame environment, does not appear in the presentation.  Your outline will be the basis for a couple of sentences of talk for each of the following questions:
% \begin{itemize}
% \item What was done?
% \item Why do it?
% \item What were the results?
% \item What do the results mean in theory and/or practice?
% \item What is the reader's benefit?
% \item How can the readers use this information for themselves? 
% \end{itemize}
% \end{abstract}


\begin{frame}{Outline}
\tableofcontents
\end{frame}



\section{definition of depth}
Using section and subsection commands, outside of frames, provides a table of contents and progress information to beamer.
\begin{frame}{definition of depth in $\mathbb{R}^p$}
\begin{defn}[statistical depth in $\mathbb{R}^p$, (Zuo and Serfling, 2000b)]
Let $\mathcal{P}$ be some class of of distributions.
The bounded and non-negative mapping $D(.,.): \mathbb{R}^p\times \mathcal{P} \rightarrow \mathbb{R}$ is called
a statistical depth function if it satisfies the following properties:
\begin{itemize}
    \item Affine invariance \\
        $D(Ax+b,P_{AX+b})=D(X,P_X)$ holds for any $\mathbb{R}^p$-valued random vector $X$,
        any $p\times p$ nonsingular matrix A and any $b\in \mathbb{R}^p$.
    \item Maximality at centre \\
        $D(\theta,P)=sup_{x\in\mathbb{R}^p}D(x,P)$ holds for any $P\in\mathcal{P}$
        having a unique centre of symmetry $\theta$ w.r.t. some notion of symmetry.
    \item Monotonicity relative to the deepest point \\
        For any $P \in \mathcal{P}$ having deepest point $\theta$, $D(x,P)\leq D(\theta+\alpha(x-\theta), P)$
        holds for all $\alpha\in[0,1]$.
    \item Vanishing at infinity
        $D(x,P)\rightarrow0$ as $||x||_{\mathbb{R}^p}\rightarrow\infty$
        for each $P\in\mathcal{P}$.
\end{itemize}
\end{defn}
\end{frame}

\begin{frame}{definition of depth in $\mathbb{R}^p$}
    (Serfling(2006)) Not necessary, but desirable property when setting D:
    \begin{itemize}
        \item Symmetry \\
            If P is symmetric about $\theta$, then so is $D(x,P)$.
        \item Continuity of D(x,P) as a function of x \\
            (or just have upper semi-continuity)
        \item Continuity of D(x,P) as a function of P
        \item Quasi-concavicity as a function of x \\
            The set $\{x:D(x,P)\geq c\}$ is convex for each real $c$.
    \end{itemize}

    \begin{exmp} [on $\mathbb{R}^1$]
        If we denote $F_P$ as cdf corresponding distribution measure $P$, then
        \begin{itemize}
            \item (By Fraiman, Muniz(2001)) $D(x,P)=1/2-[1/2-F_P(x)]$
            \item (Halfspace depth, By Tukey(1975)) $D(x,P)=min\{F_P(x), lim_{v\rightarrow x-}F_P(v)\}$
            \item (Simplical depth, By Liu(2001)) $D(x,P)=F_P(x)\{1-lim_{v\rightarrow x-} F_P(v)\}$
            \item (Modified band depth, By Cuevas, Fraiman(2009)) $D(x,P)=\frac{1}{J-1}\sum_{j=2}^J P(x\in [min(X_1,...,X_j), max(X_1,...,X_j)])$
        \end{itemize}
    \end{exmp}
\end{frame}

\begin{frame}{definition of depth in $\mathcal{F}$}
    \begin{defn}[statistical depth in $\mathcal{F}$, (Nieto-Reyes and Battey, 2016)]
        Let $(\mathcal{F},A,P)$ be probability space and $\mathcal{P}$ be class of all distribution measures on $\mathcal{F}$.
        The bounded and non-negative mapping $D(.,.): \mathcal{F}\times \mathcal{P} \rightarrow \mathbb{R}$ is called
        a statistical functional depth function if it satisfies the following properties:
        \begin{itemize}
            \item distance invariance \\
                $D(f(x),P_{f(X)})=D(X,P_X)$ for any $x\in\mathcal{F}$ and $f:\mathcal{F}\rightarrow\mathcal{F}$
                such that for any $y\in\mathcal{F}$, $d(f(x),f(y))=a_fd(x,y)$, $a_f\in\mathbb{R}-\{0\}$.
            \item Maximality at centre \\
                For any $P\in\mathcal{P}$ with unique centre of symmetry 
                $\theta$ w.r.t. some notion of symmetry, 
                $D(\theta,P)=sup_{x\in\mathcal{F}}D(x,P)$.
            \item Monotonicity (strictly decreasing) relative to the deepest point \\
                For any $P \in \mathcal{P}$ s.t. $D(z,P)=max_{x\in\mathcal{F}}D(x,P)$ exists (:deepest point $z$), 
                for $x,y\in\mathcal{F}$, $D(x,P)<D(y,P)<D(z,P)$ s.t. $min\{d(y,z),d(y,x)\}>0$ and $max\{d(y,z), d(y,x)\}<d(x,z)$.
            
        \end{itemize}
        \end{defn}
\end{frame}
\begin{frame}
    \begin{defn}[(continue.)]
        \begin{itemize}
        \item Upper semi-continuity in $x$ \\
            $D(x,P)$ is upper semi-continuous as a function of $x$.
        \item Receptivity to convex hull width across the domain. \\
            Let $C(\mathcal{F},P)$ be convex hull in $(\mathcal{F},A,P)$ defined as
            \(C(\mathcal{F},P)=\{x\in\mathcal{F} : x(v)=\alpha L(v)+ (1-\alpha)U(v), v\in V, \alpha\in[0,1]\}\)
            where $U=\{sup_{x\in E}x(v):v\in V\}$, $L=\{inf_{x\in E}x(v):v\in V\}$
            and $E$ is smallest set in $A$ s.t. $P(E)=P(\mathcal{F})$.

            Then, $D$ has a property that $D(x,P_X)<D(f(x),P_{f(X)})$ for any $x\in C(\mathcal{F},P)$
            with $D(x,P)<sup_{y\in\mathcal{F}}D(y,P)$ and $f:\mathcal{F}\rightarrow\mathcal{F}$
            s.t. $f(y(v))=\alpha(v)y(v)$ with $\alpha(v)\in(0,1)$ for all $v\in L_{\delta}$ and $\alpha(v)=1$  otherwise
            where \(L_\delta = argsup_{H\in V} \{sup_{x,y\in C(\mathcal{F},P)} d(x(H),y(H)) \leq \delta\}\)
            for any $\delta\in inf_{v\in V}d(L(v),U(v)), d(L,U)$ s.t. $\lambda(L_\delta)>0$ and $\lambda(L_\delta^c)>0$.
        \item Continuity in P \\ 
            For all $x\in\mathcal{F}$, for all $P \in \mathcal{P}$ and for every $\epsilon>0$,
            there exists a $\delta(\epsilon)>0$ s.t. $|D(x,Q)-D(x-P)|\leq\epsilon$ P-almost surely for all $Q\in\mathcal{P}$ with
            $d_P(Q,P)<\delta$ P-almost surely, where $d_P$ is metric on $\mathcal{P}$.
        \end{itemize}
    \end{defn}
\end{frame}



\begin{frame}{Check the validity of existing depth on $\mathcal{F}$}
    \begin{itemize}
        \item h-depth (Cuevas, Febrero and Fraiman(2007)) \\
        $D_h(x,P)=E_X(K_h(||x-X||_{\mathcal{L}^2[0,1]}))$ on $\mathcal{L}^2[0,1]^p$ : FTTTTT
        \item random-tukey depth (Cuesta-Albertos and Nieto-Reyes(2008))
        $D_{RT}(x,P)=min_{u\in\{u_j\}_{j=1}^k}min(P_{(u)}(-\infty,\langle u,x \rangle], P_{(u)}[\langle u,x \rangle,\infty))$ \\
        where $P_{(u)}$: marginal distribution measure of $u$, on $\mathcal{L}^2[0,1]^p$ : TTFTFT
        \item band depth(Lopez-Pintado and Romo(2009)) \\
        $D_J(x,P)=\sum_{j=2}^J P_{S_j}(x\in S_j(P))$ where $S_j(P)=\{y\in\mathcal{F} : y(v)=\alpha_1X_1(v)+...+\alpha_jX_j(v), \alpha_k\in$(j-th dim simplex),$ v\in V, X_i\sim P\}$
        on $\mathcal{C}$ with sup norm : TTFTFT
        \item modified band depth (Lopez-Pintado and Romo(2009)) \\
        $D_{MJ}(x,P)=\sum_{j=2}^J E(\lambda \{v\in V : x\in S_j(P)\})$ with above notation,
        on $\mathcal{C}$ with sup norm : TTFTFT
        \item half-region depth (Lopez-Pintado and Romo(2011)) \\
        $D_{HR}(x,P)=min\{P(X\in H_x), P(X\in E_x)\}$ where $H_x=\{y\in\mathcal{F} : y(v)\leq x(v) \text{ for all } v\in V\}$ and 
        $E_x=\{y\in\mathcal{F} : y(v)\geq x(v) \text{ for all } v\in V\}$ on $\mathcal{C}$ with sup norm : TFFTFT
        \item modified half-region depth (Lopez-Pintado and Romo(2011)) \\
        $D_{MHR}(x,P)=min\{E(\lambda\{v\in V, X(v)\leq x(v)\}), E(\lambda\{v\in V, X(v)\geq x(v)\})\}/\lambda(V)$
        on $\mathcal{C}$ with sup norm : TTFTFT
    \end{itemize}

\end{frame}

\section{Consistency of functional depth}
\begin{frame}{Consistency of functional depth: classification of existing functional depth}
    For showing consistency, classify depths to 3 groups (Stanislav Nagy(2018))
    Let D: some depth in $\mathbb{R}^p$. then
    \begin{itemize}
        \item integrated depth (Fraiman, Muniz(2001) and  Cuevas, Fraiman(2009)) \\
        form of \(FD(x,P)=\int D(f(x),f(P)d\lambda(f))\)
        \item infimal depth (Mosler(2013))\\
        form of \(ID(x,P)=inf_f D(f(x),x(P))\)
        \item band depth (Lopez-Pintado, Romo(2009))\\
        form of \(BD(x,P)=P(x\in Band(X_1,...,X_K))\) on $\mathcal{C}$
        where $Band(x_1,x_2)=\{y\in\mathcal{C}: min\{x_1(v),x_2(v)\} \leq y(v) \leq max\{x_1(v),x_2(v)\}, v\in V\}$ \\
        (extend to convex hull with many $X_i$s.)
    \end{itemize}
\end{frame}

\begin{frame}{Consistency of functional depth}
    \begin{defn}
        For given $P\in\mathcal{P}$, let $P_n\rightarrow P$ weakly.
        A functional depth $D(x,P)$ is uniformly consistent for $P$ over $\mathcal{F}$,
        if
        \[sup_{x\in\mathcal{F}}|D(x,P_n)-D(x,P)|\rightarrow 0\]
        for almost every $x$ as $n\rightarrow\infty$.
    \end{defn}
    \begin{defn}
        If $D$ is uniformly consistent for any $P\in\mathcal{P}$,
        then we say $D$ is universally consistent over $\mathcal{F}$.
    \end{defn}
    \begin{thm}[Varadarajan(1952?)]
        Let $(S,d)$ be a sparable metric space and $\mu$ be any distribution (Borel probability measure) on S.
        Then the empirical measure $\mu_n$ converges to $\mu$ almost surely:
        \[P(\{w:\mu_n(.)(w)\rightarrow\mu\})=1\]
    \end{thm}
\end{frame}

\begin{frame}{Consistency of functional depth}
    \begin{thm}[Consistency of functional band depth (Gijbels, Nagy(2015))]
        $BD(x,P)$ is not uniformly consistent over compact subset of $\mathcal{C}$.
    \end{thm}
    Possible remedy: smoothing with integration and decreasing function $w:[0,\infty)\rightarrow[0,1],w(0)=1,w(\infty)\rightarrow 0$ \\
    Adjusted band depth: \(aBD(x,P)=Ew(inf_{y\in Band(X_1,...,X_k)} ||x-y||)\) for all $x\in\mathcal{C},P\in\mathcal{P}$.
    Then, aBD is universally consistent over $\mathcal{C}$.
    \begin{thm}[Consistency of functional infimal depth (Gijbels, Nagy(2015))]
        $ID(x,P)$ is uniformly consistent over $\mathcal{C}$ for $P$ \\ 
        when $P$ is mixture of $P_1,P_2$ s.t.
        \begin{itemize}
            \item all marginal distribution of $P_1$ have continuous dist. functions.
            \item $P_2$ is concentrated in finite-dimensional subspace of $\mathcal{C}$.
        \end{itemize}
    \end{thm}
    Note that the conditions are too restrictive. (Wiener measure fails to satisfy them.)
    And it means that $ID(x,P)$ is not universally consistent over $\mathcal{C}$.
\end{frame}

\begin{frame}{Consistency of functional depth}
    \begin{thm}[Consistency of functional integrated depth (Nagy, Gijbels, Omelka, Hlubinka(2016))]
        $FD(x,P)$ is uniformly consistent over $\mathcal{C}$.
    \end{thm}
    Note that, using the definition of integration, $\mathcal{C}$ can be extend to
    Borel-measurable (may be discontinuous) functions, include $\mathcal{L}^2$.
\end{frame}


\begin{frame}{Consistency of functional depth: In practice}
    \begin{thm}[Consistency over partial observability, (Nagy,Ferraty(2018)]
        Let $P\in\mathcal{P}$ on $\mathcal{L}^2[0,1]$ and $\tilde{P_n}$ be empirical distribution of fitted $n$ curves.
        Then (under some assumptions,)
        \[sup_{x\in\mathcal{L}^2} |D(x,\tilde{P_n})-D(x,P)|\rightarrow 0\]
        almost every $x$ as $n\rightarrow\infty$ when D is adjust band depth type, h-depth type.
        If all marginal distribution of $P$ is absolutely continuous, then also true for integrated depth type.
    \end{thm}
    Proof:  \\
    step1: show $\tilde{P}_n\rightarrow P$ weakly almost every $\omega\in\Omega$ using Varadarajan theorem and with good property of fitting kernel. \\
    step2: using inner D with good convergence property, show outer D converges weakly.

\end{frame}

\begin{frame}{Consistency of functional depth: In practice}
    \begin{thm}[convergence rate of FD (Nagy,Ferraty(2018))]
        Let $P_n$ be empirical distribution of (true) n curves, and $\tilde{P_n}$ be one of fitted n curves.
        
        Suppose $P(|X(s)-X(t)|\leq L|s-t|^\beta)=1$ for all $s,t \in [0,1]$.

        Then, for any $P\in\mathcal{P}$ on $\mathcal{L}^2[0,1]$, under some conditions,
        \[sup_{x\in\mathcal{L}^2[0,1]}|FD(x,P_n)-FD(x,P)|=O_p(n^{-1/2})\]
        
        Moreover, if number of data points of $n$-th curve is comparable to $n^r$ and 
        \(sup_{v\in[0,1]} sup_{|s-s'|\leq \epsilon} |F_{(v)}(s)-F_{(v)}(s')|\leq K\epsilon^\alpha\) for some $\alpha\in(0,1]$
        where $F_{(v)}$: marginal cdf of $P$ at $v$,
        then under some conditions,
        \[sup_{x\in\mathcal{L}^2[0,1]}|FD(x,\tilde{P_n})-FD(x,P)|\]
        \[=O_p(n^{-r\alpha\beta/\{(1+\alpha)(2\beta+1)\}}) \text{ if } r<(2\beta+1)/\beta\]
        \[=O_p(\{ln(n)/n\}^{\alpha/(1+\alpha)}) \text{ if } r=(2\beta+1)/\beta\]
        \[=O_p(n^{-\alpha/(1+\alpha)}) \text{ if } r>(2\beta+1)/\beta\]
    \end{thm}
    Note that last case is dense setting, and the rate is similar to full observing case.
    In other cases, become slower.

\end{frame}





\section{Application}
\begin{frame}{Application}
More need to develop this theory and related application,
\begin{itemize}
    \item Robust and Nonparametric functional statistics \\
    procedure of with rank, nonparametric estimation of distribution, ...
    \item Exploratory Data Analysis (EDA) \\
    outlier finding, data detective works, ...
    (Center? Cluster? Symmetry? range(width)? gap(separation)? other irregularities?)
    \item classification
    if data can be classified by relation to the center.
    \item (and other things...)
\end{itemize}
\end{frame}

\end{document}
