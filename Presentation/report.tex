\ifcase 1  % choose 0=slides, 1=article, 2=refart
   \documentclass[aspectratio=169,ignorenonframetext,9pt]{beamer}
\or\documentclass[a4paper,11pt]{article}
   \usepackage{url,beamerarticle}
\or\documentclass[a4paper,11pt]{refart}
   \let\example\relax
   \usepackage{url,beamerarticle}
\fi

\ifcase 0  % choose a theme like these
    % \usetheme{boxes}
    \usetheme{Boadilla}
    % \usetheme{Goettingen}% I recommend
    % \usetheme{Singapore}
    % \usetheme{Pittsburgh}
    % \usetheme{Madrid}
    % \usetheme{Warsaw} % common choice, but often poor
\fi

\usepackage{graphicx,pgfplots,parskip}
\usepackage{amsmath,amsfonts,amssymb,amsthm,epsfig,epstopdf,url,array}



\theoremstyle{plain}
\newtheorem{thm}{Theorem}[section]
\newtheorem{lem}[thm]{Lemma}
\newtheorem{prop}[thm]{Proposition}

\theoremstyle{definition}
\newtheorem{defn}{Definition}[section]
\newtheorem{conj}{Conjecture}[section]
\newtheorem{exmp}{Example}[section]


\title{Functional depths}
\author{Choi Seokjun}
\date{18 Dec. 2019}


\begin{document}

\begin{frame}
\maketitle
\end{frame}


% \begin{abstract}
% This abstract, being outside the frame environment, does not appear in the presentation.  Your outline will be the basis for a couple of sentences of talk for each of the following questions:
% \begin{itemize}
% \item What was done?
% \item Why do it?
% \item What were the results?
% \item What do the results mean in theory and/or practice?
% \item What is the reader's benefit?
% \item How can the readers use this information for themselves? 
% \end{itemize}
% \end{abstract}



\tableofcontents



\section{Definition of the depth}

\begin{frame}{definition of depth in $\mathcal{R}^p$}
As preliminary, I firstly start with introducing a definition of statistical depth on $\mathbb{R}^p$.


\begin{defn}[statistical depth in $\mathbb{R}^p$, (Zuo and Serfling, 2000b)]
Let $\mathcal{P}$ be some class of of distributions.
The bounded and non-negative mapping $D(.,.): \mathbb{R}^p\times \mathcal{P} \rightarrow \mathbb{R}$ is called
a statistical depth function if it satisfies the following properties:
\begin{itemize}
    \item Affine invariance \\
        $D(Ax+b,P_{AX+b})=D(X,P_X)$ holds for any $\mathbb{R}^p$-valued random vector $X$,
        any $p\times p$ nonsingular matrix A and any $b\in \mathbb{R}^p$.
    \item Maximality at center \\
        $D(\theta,P)=sup_{x\in\mathbb{R}^p}D(x,P)$ holds for any $P\in\mathcal{P}$
        having a unique center of symmetry $\theta$ w.r.t. some notion of symmetry.
    \item Monotonicity relative to the deepest point \\
        For any $P \in \mathcal{P}$ having deepest point $\theta$, $D(x,P)\leq D(\theta+\alpha(x-\theta), P)$
        holds for all $\alpha\in[0,1]$.
    \item Vanishing at infinity
        $D(x,P)\rightarrow0$ as $||x||_{\mathbb{R}^p}\rightarrow\infty$
        for each $P\in\mathcal{P}$.
\end{itemize}
\end{defn}


Any function $D(.,.)$ that satisfies above properties can be used as statistical depth in $\mathbb{R}^p$.
I note that, although the definition is made by using population distribution measure but
in practice we do not know about true one, we can replace it by empirical distribution measure when $n$ curves observed,
(denote $P_n$) and can calculate sample depth from that empirical distribution. 
For validity to this replacement, see below part dealing with consistency.

Next, belows are not necessary, but desirable property when setting statistical depth function D:
suggested by Serfling(2006). It is notable because it shows some properties of statistical depth more intuitively than definition does.
\begin{itemize}
    \item Symmetry \\
        If P is symmetric about $\theta$, then so is $D(x,P)$.
    \item Continuity of D(x,P) as a function of x \\
        (or just have upper semi-continuity)
    \item Continuity of D(x,P) as a function of P
    \item Quasi-concavicity as a function of x \\
        The set $\{x:D(x,P)\geq c\}$ is convex for each real $c$.
\end{itemize}

I give some examples about some 'toy' depth in $\mathbb{R}^1$.
\begin{exmp} [on $\mathbb{R}^1$]
    If we denote $F_P$ as cdf corresponding distribution measure $P$, then
    \begin{itemize}
        \item (By Fraiman, Muniz(2001))
        \[D(x,P)=1/2-[1/2-F_P(x)]\]
        \item (Halfspace depth, By Tukey(1975))
        \[D(x,P)=min\{F_P(x), lim_{v\rightarrow x-}F_P(v)\}\]
        \item (Simplical depth, By Liu(2001))
        \[D(x,P)=F_P(x)\{1-lim_{v\rightarrow x-} F_P(v)\}\]
        \item (Modified band depth, By Cuevas, Fraiman(2009))
        \[D(x,P)=\frac{1}{J-1}\sum_{j=2}^J P(x\in [min(X_1,...,X_j), max(X_1,...,X_j)])\]
    \end{itemize}
\end{exmp}
\end{frame}

\begin{frame}{definition of depth in $\mathcal{F}$}
Now, I will introduce the definition of the functional depth, as one of the latest expansions of ordinary depth
suggested by Nieto-Reyes and Battey.
\begin{defn}[statistical depth in $\mathcal{F}$, (Nieto-Reyes and Battey, 2016)]
    Let $(\mathcal{F},A,P)$ be probability space and $\mathcal{P}$ be class of all distribution measures on $\mathcal{F}$,
    $d$ be metric on $\mathcal{F}$.
    The bounded and non-negative mapping $D(.,.): \mathcal{F}\times \mathcal{P} \rightarrow \mathbb{R}$ is called
    a statistical functional depth function if it satisfies the following properties:
    \begin{enumerate}
        \item distance invariance \\
            $D(f(x),P_{f(X)})=D(X,P_X)$ for any $x\in\mathcal{F}$ and $f:\mathcal{F}\rightarrow\mathcal{F}$
            such that for any $y\in\mathcal{F}$, $d(f(x),f(y))=a_fd(x,y)$, $a_f\in\mathbb{R}-\{0\}$.
        \item Maximality at center \\
            For any $P\in\mathcal{P}$ with unique center of symmetry 
            $\theta$ w.r.t. some notion of symmetry, 
            $D(\theta,P)=sup_{x\in\mathcal{F}}D(x,P)$.
        \item Monotonicity (strictly decreasing) relative to the deepest point \\
            For any $P \in \mathcal{P}$ s.t. $D(z,P)=max_{x\in\mathcal{F}}D(x,P)$ exists (:deepest point $z$), 
            for $x,y\in\mathcal{F}$, $D(x,P)<D(y,P)<D(z,P)$ s.t. $min\{d(y,z),d(y,x)\}>0$ and $max\{d(y,z), d(y,x)\}<d(x,z)$.
        \item Upper semi-continuity in $x$ \\
            $D(x,P)$ is upper semi-continuous as a function of $x$.
        \item Receptivity to convex hull width across the domain. \\
            Let $C(\mathcal{F},P)$ be convex hull in $(\mathcal{F},A,P)$ defined as
            \(C(\mathcal{F},P)=\{x\in\mathcal{F} : x(v)=\alpha L(v)+ (1-\alpha)U(v), v\in V, \alpha\in[0,1]\}\)
            where $U=\{sup_{x\in E}x(v):v\in V\}$, $L=\{inf_{x\in E}x(v):v\in V\}$
            and $E$ is smallest set in $A$ s.t. $P(E)=P(\mathcal{F})$.

            Then, $D$ has a property that $D(x,P_X)<D(f(x),P_{f(X)})$ for any $x\in C(\mathcal{F},P)$
            with $D(x,P)<sup_{y\in\mathcal{F}}D(y,P)$ and $f:\mathcal{F}\rightarrow\mathcal{F}$
            s.t. $f(y(v))=\alpha(v)y(v)$ with $\alpha(v)\in(0,1)$ for all $v\in L_{\delta}$ and $\alpha(v)=1$  otherwise
            where \(L_\delta = argsup_{H\in V} \{sup_{x,y\in C(\mathcal{F},P)} d(x(H),y(H)) \leq \delta\}\)
            for any $\delta\in inf_{v\in V}d(L(v),U(v)), d(L,U)$ s.t. $\lambda(L_\delta)>0$ and $\lambda(L_\delta^c)>0$.
        \item Continuity in P \\ 
            For all $x\in\mathcal{F}$, for all $P \in \mathcal{P}$ and for every $\epsilon>0$,
            there exists a $\delta(\epsilon)>0$ s.t. $|D(x,Q)-D(x-P)|\leq\epsilon$ P-almost surely for all $Q\in\mathcal{P}$ with
            $d_P(Q,P)<\delta$ P-almost surely, where $d_P$ is metric on $\mathcal{P}$.
        \end{enumerate}
    \end{defn}

I should give an caveat. The definition of functional depth is being studied for now, so
there are still different views about it and many researchers are trying to
suggest better property bundle for it. So, above definition are not concrete, not absolute-unique thing.

For other suggestion, next ones are often viewed :
\begin{itemize}
    \item 'convex depth level set'(ex. Narisetty and Nair, 2015)
    \item 'null at the boundary'(or, similarly 'Vanishing at infinity')(Mosler and Polyakov, 2012)
    \item 'non-degeneracy with gaussian process class' (Chakraborty and Chaudhuri, 2014b)
    \item 'maximal value at 0 on some Gaussian-type processes (or curves).
\end{itemize}

Additionally, I comment for each property of above definition.
\begin{enumerate}
    \item It is direct expansion of 'Affine invariance' from the definition of depth on $\mathbb{R}^p$.
    It means that depth should be invariant with transformation without changing centrality.
    Especially, with just rescaled distance metric.
    
    \item It is also direct expansion of 'Maximality at center' the definition on $\mathbb{R}^p$'s.
    Since we are working in metric space, the condition is rewritten like 
    some form of triangle inequality using distance.
    And I note that, it implies 'Maximality at 0 with gaussian process' of one item of other suggestion.

    But, if strictly speaking, in general the notion - the center of the distribution on function space - has much ambiguousness itself,
    especially if we consider other distributions except gaussian or trivial ones with properties like stationary, having obvious mean, symmetric, continuous, etc.
    So, We can say that the definition above has similar problem.
    
    \item It is expansion of 'monotonicity relative to the deepest point' and 'vanishing at infinity'.
    In fact, the strictness is too restrictive. If there isn't, it does not violation of the notion of the depth.
    
    But, in practice, because some suggested depths in past have some 'degeneracy problem': all curves have
    same depth despite having different level of centrality at some specific distributions,
    this strictness may help to exclude them that make a situation like degeneracy.

    And, the strictness can be used for more efficient application like classification.
    Moreover, with second property, this property implies 'null at boundary'.

    \item This property is for relation between the depth and the distribution measure, especially 
    cumulative distribution function. Since cdf has upper-semi continuity property,
    for using the depth to estimate the cdf, the depth have better to have this property, too.

    \item This is property that Nieto-Reyes and Battey, who suggest this definition, add newly. 
    Their intention is, they want for the depth to have some robustness 
    that having low influence level in their depth of curve with 'small difference',
    when some curves are varying in part of domain which data exhibit little variability, 
    
    But I think the mathematical expression of the condition has too restrictive part,
    when think the validity of strict inequality, considering to their intention.

    \item It is for consistency of the depth. When empirical distribution to true distribution,
    if their depth is not converge, it may be ridiculous thing. For excluding these situation,
    we set proper depth function having property that converges well when both cases that 
    the number of the curves is larger and the number of data points grows up.
    
\end{enumerate}

\end{frame}



\begin{frame}{Check the validity of existing depth on $\mathcal{F}$}
Below are some 'classical' functional depths.
(They are frequently cited at other authors' papers 
for comparing suggested depth by author of itself.)
Bold letters indicates whether each depth satisfying above 1-6 properties respectively.

For notation, $\mathcal{L}^2$ means the space of square integrable function with its norm,
and $\mathcal{C}$ means the space of continuous function.
Note that, $\mathcal{C}$ equipping sup norm becomes a complete normed vector space (Banach space)
without inner product, but it is not problem because above definition is on metric space
which is more general setting than inner product space or normed space.

\begin{itemize}
    \item h-depth (Cuevas, Febrero and Fraiman(2007)) : \textbf{FTTTTT}\\
    $D_h(x,P)=E_X(K_h(||x-X||_{\mathcal{L}^2[0,1]}))$ on $\mathcal{L}^2[0,1]^p$ 
    \item random-tukey depth (Cuesta-Albertos and Nieto-Reyes(2008)) : \textbf{TTFTFT}\\
    $D_{RT}(x,P)=min_{u\in\{u_j\}_{j=1}^k}min(P_{(u)}(-\infty,\langle u,x \rangle], P_{(u)}[\langle u,x \rangle,\infty))$ \\
    where $P_{(u)}$: marginal distribution measure of $u$, on $\mathcal{L}^2[0,1]^p$
    \item band depth(Lopez-Pintado and Romo(2009)) : \textbf{TTFTFT}\\
    $D_J(x,P)=\sum_{j=2}^J P_{S_j}(x\in S_j(P))$ where $S_j(P)=\{y\in\mathcal{F} : y(v)=\alpha_1X_1(v)+...+\alpha_jX_j(v), \alpha_k\in$(j-th dim simplex),$ v\in V, X_i\sim P\}$
    on $\mathcal{C}$ with sup norm 
    \item modified band depth (Lopez-Pintado and Romo(2009)) : \textbf{TTFTFT}\\
    $D_{MJ}(x,P)=\sum_{j=2}^J E(\lambda \{v\in V : x(v)\in S_j(P)\})$ with above notation,
    on $\mathcal{C}$ with sup norm 
    \item half-region depth (Lopez-Pintado and Romo(2011)) : \textbf{TFFTFT}\\
    $D_{HR}(x,P)=min\{P(X\in H_x), P(X\in E_x)\}$ where $H_x=\{y\in\mathcal{F} : y(v)\leq x(v) \text{ for all } v\in V\}$ and 
    $E_x=\{y\in\mathcal{F} : y(v)\geq x(v) \text{ for all } v\in V\}$ on $\mathcal{C}$ with sup norm 
    \item modified half-region depth (Lopez-Pintado and Romo(2011)) : \textbf{TTFTFT}\\
    $D_{MHR}(x,P)=min\{E(\lambda\{v\in V, X(v)\leq x(v)\}), E(\lambda\{v\in V, X(v)\geq x(v)\})\}/\lambda(V)$
    on $\mathcal{C}$ with sup norm
\end{itemize}

\end{frame}

\section{Consistency of functional depth}
For showing consistency, I firstly follow the classification of depths to 3 groups by Stanislav Nagy(2018).

From this line, let D be some depth in $\mathbb{R}^p$. then
\begin{itemize}
    \item integrated depth (Fraiman, Muniz(2001) and  Cuevas, Fraiman(2009)) \\
    form of \(FD(x,P)=\int D(f(x),f(P)d\lambda(f))\)
    \item infimal depth (Mosler(2013))\\
    form of \(ID(x,P)=inf_f D(f(x),f(P))\)
    \item band depth (Lopez-Pintado, Romo(2009))\\
    form of \(BD(x,P)=P(x\in Band(X_1,...,X_K))\) on $\mathcal{C}$
    where $Band(x_1,x_2)=\{y\in\mathcal{C}: min\{x_1(v),x_2(v)\} \leq y(v) \leq max\{x_1(v),x_2(v)\}, v\in V\}$ \\
    (extend to convex hull with many $X_i$s.)
\end{itemize}
These classification do NOT embrace all depths suggested until now, but I'll only focus above three-type-case
for simple discussion I do below.


\begin{frame}{Consistency of functional depth}
There are some notions about consistency for statistical depths.

\begin{defn}
    For given $P\in\mathcal{P}$, let $P_n\rightarrow P$ weakly.
    A functional depth $D(x,P)$ is uniformly consistent for $P$ over $\mathcal{F}$,
    if
    \[sup_{x\in\mathcal{F}}|D(x,P_n)-D(x,P)|\rightarrow 0\]
    for almost every $x$ as $n\rightarrow\infty$.
\end{defn}
\begin{defn}
    If $D$ is uniformly consistent for any $P\in\mathcal{P}$,
    then we say $D$ is universally consistent over $\mathcal{F}$.
\end{defn}
    
The consistency theorems depend on below theorem, guaranteeing that
empirical distribution from samples converges to true population distribution.
\begin{thm}[Varadarajan(1956)]
    Let $(S,d)$ be a sparable metric space and $\mu$ be any distribution (Borel probability measure) on S.
    Then the empirical measure $\mu_n$ converges to $\mu$ almost surely:
    \[P(\{w:\mu_n(.)(w)\rightarrow\mu\})=1\]
\end{thm}
The proof can be found by many books about probability theory, like 'Real analysis and probability' by Dudley.

I'll introduce only the statements of consistency theorems without proof,
since they are too long to shift all thing to here. If need, see of each paper's.

\begin{thm}[Consistency of functional band depth (Gijbels, Nagy(2015))]
    $BD(x,P)$ is not uniformly consistent over compact subset of $\mathcal{C}$.
\end{thm}

Because the band type depths are naturally expanded from multi-finite dimensional one and 
use very intuitive notion, Many researchers try to adjust it to satisfy consistency.

One possible remedy is here: smoothing with integration and decreasing function $w:[0,\infty)\rightarrow[0,1],w(0)=1,w(\infty)\rightarrow 0$ \\
Adjusted band depth: \(aBD(x,P)=Ew(inf_{y\in Band(X_1,...,X_k)} ||x-y||)\) for all $x\in\mathcal{C},P\in\mathcal{P}$.
Then, aBD is universally consistent over $\mathcal{C}$. But as cost, lose some simplicity which original one has.

Next, here is the consistency theorem for infimal type depths.
\begin{thm}[Consistency of functional infimal depth (Gijbels, Nagy(2015))]
    $ID(x,P)$ is uniformly consistent over $\mathcal{C}$ for $P$ \\ 
    when $P$ is mixture of $P_1,P_2$ s.t.
    \begin{itemize}
        \item all marginal distribution of $P_1$ have continuous dist. functions.
        \item $P_2$ is concentrated in finite-dimensional subspace of $\mathcal{C}$.
    \end{itemize}
\end{thm}
Note that the conditions are too restrictive. We can observe that $ID(x,P)$ is not universally consistent over $\mathcal{C}$.
What is worse, even Wiener measure (Gaussian measure) fails to satisfy them.
To consider intuitively, see Brownian motion processes start at 0. Then they are
same depth (same boundary level!) regardless of each of behavior at rear.
And this degeneracy cannot be solved even if the number of curves become larger.

Finally, this is the consistency theorem for integrated type depths.
\begin{thm}[Consistency of functional integrated depth (Nagy, Gijbels, Omelka, Hlubinka(2016))]
        $FD(x,P)$ is uniformly consistent over $\mathcal{C}$.
\end{thm}
Note that, using the definition of integration, $\mathcal{C}$ can be extend to
Borel-measurable (may be discontinuous) functions, include $\mathcal{L}^2$.
\end{frame}


\begin{frame}{Consistency of functional depth: In practice}
Since in practice we can get only finite data points for each curves, we need two direction's convergence property. 
The first is of the direction of the number of curves grows, and the second is of the direction of the number of data points grows for each curve.
It is so complicate to deal with, so there are a little proofs of consistency for only special types of depths, but not for general depths.
Below theorem is only for adjust band depth type, h-depth type and integrated type depth.

\begin{thm}[Consistency over partial observability, (Nagy, Ferraty(2018)]
    Let $P\in\mathcal{P}$ on $\mathcal{L}^2[0,1]$ and $\tilde{P_n}$ be empirical distribution of fitted $n$ curves.
    (either approximated by using kernel smoothing or linear combination of basis on separable spaces.)
    Then (under some assumptions,)
    \[sup_{x\in\mathcal{L}^2} |D(x,\tilde{P_n})-D(x,P)|\rightarrow 0\]
    almost every $x$ as $n\rightarrow\infty$ when D is adjust band depth type, h-depth type.
    If all marginal distribution of $P$ is absolutely continuous, then also true for integrated depth type.
\end{thm}

The 'some conditions' are too technical for purpose to get intuition, so I intentionally skip them.
But I should note that they include some conditions on smoothing kernel which is used for fit to curve from data points,
and they make the depth value calculated on approximated curves converge true depth value on population's distribution.
If want, see 'Nagy, S., \& Ferraty, F. (2019). Data depth for measurable noisy random functions. Journal of Multivariate Analysis, 170, 95-114.'.
The proof is proceeded 2 steps. I briefly preview these.
\begin{enumerate}
    \item show $\tilde{P}_n\rightarrow P$ weakly almost every $\omega\in\Omega$ using Varadarajan theorem and some good properties of fitting kernel.
    \item using the convergence property of inner D (in $\mathbb{R}^p$),
    show that expanded $D$ to $\mathcal{F}$ also converges weakly to true $P$ together with the specific form of the depth.
\end{enumerate}

Lastly, I give some convergence rates that are proved lately by some researchers.
\begin{thm}[convergence rate of FD (Nagy,Ferraty(2018))]
    Let $P_n$ be empirical distribution of (true) n curves, and $\tilde{P_n}$ be one of fitted n curves.

    Suppose $P(|X(s)-X(t)|\leq L|s-t|^\beta)=1$ for all $s,t \in [0,1]$.

    Then, for any $P\in\mathcal{P}$ on $\mathcal{L}^2[0,1]$, under some conditions,
    \[sup_{x\in\mathcal{L}^2[0,1]}|FD(x,P_n)-FD(x,P)|=O_p(n^{-1/2})\]
    
    Moreover, if number of data points of $n$-th curve is comparable to $n^r$ and 
    \(sup_{v\in[0,1]} sup_{|s-s'|\leq \epsilon} |F_{(v)}(s)-F_{(v)}(s')|\leq K\epsilon^\alpha\) for some $\alpha\in(0,1]$
    where $F_{(v)}$: marginal cdf of $P$ at $v$,
    then under some conditions,
    \[sup_{x\in\mathcal{L}^2[0,1]}|FD(x,\tilde{P_n})-FD(x,P)|\]
    \[=O_p(n^{-r\alpha\beta/\{(1+\alpha)(2\beta+1)\}}) \text{ if } r<(2\beta+1)/\beta\]
    \[=O_p(\{ln(n)/n\}^{\alpha/(1+\alpha)}) \text{ if } r=(2\beta+1)/\beta\]
    \[=O_p(n^{-\alpha/(1+\alpha)}) \text{ if } r>(2\beta+1)/\beta\]
\end{thm}
Note that last case means some similar situation of the dense setting case. 
And we can observe that the rate is similar to full observing case.
(think $\alpha=1$ case: the marginal distribution's convergence is fast enough.)
In other cases such as case of the sparse data points, the convergence becomes slower.

\end{frame}





\section{Application}
\begin{frame}{Application}
\begin{itemize}
    \item Median estimation
    \item Robust and Nonparametric functional statistics \\
    procedure of with rank (eg. rank test on functional data), nonparametric estimation of distribution or summary statistic, ...
    \item Exploratory Data Analysis (EDA) \\
    outlier detection, data expression (ex. functional box plot), ...\\
    (Center? Cluster? Symmetry? range(width)? gap(separation)? other irregularities?)
    \item classification \\
    when data can be classified by relation to the center. \\
    (if needed, after some transformation)
    \item (and other things...)
\end{itemize}

Now there are many researchers that broaden or newly opening analysis of functional data using depth.
And right now, another applications are being developed in real time.
So, I expect that the notion of the functional depth will be used wider at even near future.

Finally note that, the Usability in application yields some other criteria about defining and comparing depth
such as computational advantage or the validity of central region or value (besides order) of specific depth.
For example, a classification problem as a crucial application of depth, the latter is so important.
For this reason, there is the trend when suggesting a new depth, not only a theoretical property, 
but also an applicability also become more to refer.

\end{frame}

\section{Reference}
\begin{frame}{main reference}
    \begin{itemize}
        \item Nieto-Reyes, A., \& Battey, H. (2016). A topologically valid definition of depth for functional data. Statistical Science, 31(1), 61-79.
        \item Nagy, S., \& Ferraty, F. (2019). Data depth for measurable noisy random functions. Journal of Multivariate Analysis, 170, 95-114.
    \end{itemize}
\end{frame}

\begin{frame}{other reference}
    \begin{itemize}
        \item Fraiman, R., \& Muniz, G. (2001). Trimmed means for functional data. Test, 10(2), 419-440.
        \item Cuevas, A., Febrero, M., \& Fraiman, R. (2007). Robust estimation and classification for functional data via projection-based depth notions. Computational Statistics, 22(3), 481-496.
        \item Cuevas, A., \& Fraiman, R. (2009). On depth measures and dual statistics. A methodology for dealing with general data. Journal of Multivariate Analysis, 100(4), 753-766.
        \item Lopez-Pintado, S., \& Romo, J. (2009). On the concept of depth for functional data. Journal of the American Statistical Association, 104(486), 718-734.
        \item Lopez-Pintado, S., \& Romo, J. (2011). A half-region depth for functional data. Computational Statistics \& Data Analysis, 55(4), 1679-1695.
        \item Long, J. P., \& Huang, J. Z. (2015). A study of functional depths. arXiv preprint arXiv:1506.01332.
        \item Gijbels, I., \& Nagy, S. (2015). Consistency of non-integrated depths for functional data. Journal of Multivariate Analysis, 140, 259-282.
        \item Narisetty, N. N., \& Nair, V. N. (2016). Extremal depth for functional data and applications. Journal of the American Statistical Association, 111(516), 1705-1714.
        \item Hubert, M., Rousseeuw, P., \& Segaert, P. (2017). Multivariate and functional classification using depth and distance. Advances in Data Analysis and Classification, 11(3), 445-466.
    \end{itemize}
\end{frame}

\begin{frame}{book reference}
    \begin{itemize}
        \item Dudley, R.M., Real analysis and Probability, (1989). Cambridge. ISBN 978-0521007542
        \item Aneiros, G., Bongiorno, E.G., Cao, R., \& Vieu, P., Functional statistics and related field (2017). Springer. ISBN 978-3-319-55846-2
    \end{itemize}
\end{frame}

\end{document}
